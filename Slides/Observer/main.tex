\documentclass{beamer}

\pdfmapfile{+sansmathaccent.map}


\mode<presentation>
{
  \usetheme{Warsaw} % or try Darmstadt, Madrid, Warsaw, Rochester, CambridgeUS, ...
  \usecolortheme{seahorse} % or try seahorse, beaver, crane, wolverine, ...
  \usefonttheme{serif}  % or try serif, structurebold, ...
  \setbeamertemplate{navigation symbols}{}
  \setbeamertemplate{caption}[numbered]
} 


%%%%%%%%%%%%%%%%%%%%%%%%%%%%
% itemize settings

\definecolor{mypink}{RGB}{255, 30, 80}
\definecolor{mydarkblue}{RGB}{60, 160, 255}
\definecolor{myblue}{RGB}{240, 240, 255}
\definecolor{mygreen}{RGB}{0, 200, 0}
\definecolor{mygreen2}{RGB}{245, 255, 230}
\definecolor{mygray}{gray}{0.8}

\setbeamertemplate{itemize items}[default]

\setbeamertemplate{itemize item}{\color{mygreen}$\blacksquare$}
\setbeamertemplate{itemize subitem}{\color{mydarkblue}$\blacktriangleright$}
\setbeamertemplate{itemize subsubitem}{\color{mygray}$\blacksquare$}



\setbeamercolor{palette quaternary}{fg=white,bg=mydarkblue}
\setbeamercolor{titlelike}{parent=palette quaternary}

\setbeamercolor{palette quaternary2}{fg=black,bg=myblue}
\setbeamercolor{frametitle}{parent=palette quaternary2}



\setbeamerfont{frametitle}{size=\Large,series=\scshape}
\setbeamerfont{framesubtitle}{size=\normalsize,series=\upshape}





%%%%%%%%%%%%%%%%%%%%%%%%%%%%
% block settings

\setbeamercolor{block title}{bg=red!30,fg=black}

\setbeamercolor*{block title example}{bg=mygreen!40!white,fg=black}

\setbeamercolor*{block body example}{fg= black,
bg= mygreen2}


%%%%%%%%%%%%%%%%%%%%%%%%%%%%
% URL settings
\hypersetup{
    colorlinks=true,
    linkcolor=blue,
    filecolor=blue,      
    urlcolor=blue,
}

%%%%%%%%%%%%%%%%%%%%%%%%%%

\renewcommand{\familydefault}{\rmdefault}

\usepackage{amsmath}
\usepackage{mathtools}


\usepackage{subcaption}

\newcommand{\bo}[1] {\mathbf{#1}}
\newcommand{\R} {\mathbb{R}}


%%%%%%%%%%%%%%%%%%%%%%%%%%%%
% code settings

\usepackage{listings}
\usepackage{color}
% \definecolor{mygreen}{rgb}{0,0.6,0}
% \definecolor{mygray}{rgb}{0.5,0.5,0.5}
\definecolor{mymauve}{rgb}{0.58,0,0.82}
\lstset{ 
  backgroundcolor=\color{white},   % choose the background color; you must add \usepackage{color} or \usepackage{xcolor}; should come as last argument
  basicstyle=\footnotesize,        % the size of the fonts that are used for the code
  breakatwhitespace=false,         % sets if automatic breaks should only happen at whitespace
  breaklines=true,                 % sets automatic line breaking
  captionpos=b,                    % sets the caption-position to bottom
  commentstyle=\color{mygreen},    % comment style
  deletekeywords={...},            % if you want to delete keywords from the given language
  escapeinside={\%*}{*)},          % if you want to add LaTeX within your code
  extendedchars=true,              % lets you use non-ASCII characters; for 8-bits encodings only, does not work with UTF-8
  firstnumber=0000,                % start line enumeration with line 0000
  frame=single,	                   % adds a frame around the code
  keepspaces=true,                 % keeps spaces in text, useful for keeping indentation of code (possibly needs columns=flexible)
  keywordstyle=\color{blue},       % keyword style
  language=Octave,                 % the language of the code
  morekeywords={*,...},            % if you want to add more keywords to the set
  numbers=left,                    % where to put the line-numbers; possible values are (none, left, right)
  numbersep=5pt,                   % how far the line-numbers are from the code
  numberstyle=\tiny\color{mygray}, % the style that is used for the line-numbers
  rulecolor=\color{black},         % if not set, the frame-color may be changed on line-breaks within not-black text (e.g. comments (green here))
  showspaces=false,                % show spaces everywhere adding particular underscores; it overrides 'showstringspaces'
  showstringspaces=false,          % underline spaces within strings only
  showtabs=false,                  % show tabs within strings adding particular underscores
  stepnumber=2,                    % the step between two line-numbers. If it's 1, each line will be numbered
  stringstyle=\color{mymauve},     % string literal style
  tabsize=2,	                   % sets default tabsize to 2 spaces
  title=\lstname                   % show the filename of files included with \lstinputlisting; also try caption instead of title
}

%%%%%%%%%%%%%%%%%%%%%%%%%%%%
% tikz settings

\usepackage{tikz}
\tikzset{every picture/.style={line width=0.75pt}}


\title{Observers}
\subtitle{Control Theory, Lecture 10}
\author{by Sergei Savin}
\centering
\date{Spring 2021}



\begin{document}
\maketitle


\begin{frame}{Content}
\begin{itemize}
\item Hamilton-Jacobi-Bellman equation
% \begin{itemize}
%     \item Definitions
%     \item Cost, optimal cost
%     \item Differentiating optimal cost
% \end{itemize}
% \item Algebraic Riccati equation
% \begin{itemize}
%     \item HJB for LTI
%     \item Linear Quadratic Regulator
%     \item Numerical methods
% \end{itemize}
\end{itemize}
\end{frame}




\begin{frame}{Measurement}
\framesubtitle{How do we know the state?}
\begin{flushleft}

Before we considered systems and control laws of the following type:

\begin{equation}
\begin{cases}
\dot {\bo{x}} = \bo{A} \bo{x} + \bo{B} \bo{u}\\
\bo{u} = \mathbf K \bo{x}
\end{cases}
\end{equation}

But when we implement that control law, how do we know the current value of $\bo{x}$? Previously we always took it from simulation. 

\bigskip

In practice, we take it from \emph{measurement}.

\end{flushleft}
\end{frame}

\begin{frame}{Measurement}
\framesubtitle{Why information is imperfect?}
\begin{flushleft}

There are a number of reasons why we can not directly measure the state of the system. Here are some:

\begin{itemize}
\item Digital measurements are done in discrete time intervals;
\item Unpredicted events (faults, collisions, etc.);
\item Un-modelled kinematics or dynamics (links bending, gear box backlash,  friction, etc.) making the very definition of the state disconnected from reality;
\item Lack of sensors;
\item Imprecise, nonlinear and biased sensors;
\item Physics, quantum-scale effects and alike;
\end{itemize}

\end{flushleft}
\end{frame}

\begin{frame}{Measurement}
\framesubtitle{Definition}
\begin{flushleft}

Let us introduce new notation. Assume we have an LTI system of the following form:

\begin{equation}
\begin{cases}
\dot {\bo{x}} = \bo{A} \bo{x} + \bo{B} \bo{u} \\
\bo{y} = \bo{C} \bo{x} \\
\bo{u} = \bo{g}(\hat{\bo{x}}, t)\\
\hat{\bo{x}} = \bo{o}(\cdot)
\end{cases}
\end{equation}

Then:

\begin{itemize}
\item $\bo{x}$ is the state (actual or true state)
\item $\bo{y}$ is the output (actual or true output)
% \item $\Tilde{\bo{y}}$ is the measured output
\item $\hat{\bo{x}}$ is the estimated (observed) state
\item $\hat{\bo{y}}$ is the estimated (observed) output
\end{itemize}

Notice that we never know true state $\bo{x}$, and therefore for the control purposes we have to use the estimated state $\hat{\bo{x}}$.

\end{flushleft}
\end{frame}



\begin{frame}{Observation}
\framesubtitle{Using the knowledge about dynamics}
\begin{flushleft}

Let us consider autonomous dynamical system
\begin{equation}
\label{eq:LTI}
\begin{cases}
\dot {\bo{x}} = \bo{A} \bo{x} + \bo{B} \bo{u} \\
\bo{y} = \bo{C} \bo{x}
\end{cases}
\end{equation}
%
with measurements $\bo{y}$. We want to get as good estimate of the state $\hat{\bo{x}}$ as we can.

\bigskip

First note: dynamics should also hold for our observed state:
\begin{equation}
\hat{\dot {\bo{x}}} = \bo{A} \hat{\bo{x}} + \bo{B} \bo{u}
\end{equation}
%
Therefore if we know the initial conditions of our system exactly, and we know our model exactly, we can find exact state of the system without using measurement $\Tilde{\mathbf y}$. 

This we can call open loop observation. Unfortunately, we never know neither model nor initial conditions exactly.


\end{flushleft}
\end{frame}





\begin{frame}{Observation}
\framesubtitle{Observer}
\begin{flushleft}

We propose \emph{observer} that takes into account measurements in a linear way:

\begin{equation}
\label{eq:Observer}
\hat{\dot {\mathbf x}} = \mathbf A \hat{\mathbf x} + \mathbf B \mathbf u + \mathbf L(\mathbf y - \bo{C} \hat{\mathbf x})
\end{equation}
%
with measurements $\bo{y}$. With this observer, we want to get as good estimate of the state $\hat{\bo{x}}$ as we can.

\bigskip

Let's define state estimation error as $\varepsilon = \hat{\mathbf x} - \mathbf x$. We can subtract \eqref{eq:LTI} from \eqref{eq:Observer}, to get \emph{observer error dynamics}:

\begin{equation}
\hat{\dot {\mathbf x}} - \dot {\mathbf x}= 
\mathbf A \hat{\mathbf x} - \mathbf A \mathbf x + 
\mathbf L(\mathbf y - \bo{C} \hat{\mathbf x})
\end{equation}

\begin{equation}
\dot {\varepsilon}= 
(\mathbf A - \mathbf L \bo{C}) \varepsilon
\end{equation}

\end{flushleft}
\end{frame}





\begin{frame}{Observation}
\framesubtitle{Observer gains}
\begin{flushleft}

The observer $\dot {\varepsilon}= 
(\mathbf A - \mathbf L \bo{C}) \varepsilon$ is \emph{stable} (i.e., the state estimation error tends to zero), as long as the following matrix is negative-definite:

\[
\mathbf A - 
\mathbf L \mathbf C < 0
\]

We need to find $\mathbf L$. Let us observe the key difference between observer design and controller design:

\bigskip

\begin{itemize}
    \item Controller design: find such $\mathbf K$ that $\mathbf A - \mathbf B \mathbf K < 0$.
    \item Observer design: find such $\mathbf L$ that: $\mathbf A - \mathbf L \mathbf C < 0$
\end{itemize}

\bigskip

We have instruments for finding $\mathbf K$, what about $\mathbf L$?

\end{flushleft}
\end{frame}


\begin{frame}{Observer Design}
\framesubtitle{General case: design via Riccati eq.}
\begin{flushleft}

In general, we can observe that if $\mathbf A - \mathbf L \mathbf C$ is negative-definite, then $(\mathbf A - 
\mathbf L \mathbf C)^{\top}$ is negative-definite too (by definition of the negative-definiteness). 

\bigskip

Therefore, we can solve the following \emph{dual problem}:

\begin{itemize}
    \item find such $\mathbf L$ that $\mathbf A^{\top} - 
\mathbf C^{\top} \mathbf L^{\top} < 0$.
\end{itemize}

\bigskip

The dual problem is \emph{equivalent} to the control design problem. We can solve it by producing and solving algebraic Riccati equation, as in the LQR formulation. In pseudo-code it can be represented the following way:

\bigskip

$\mathbf L^{\top}$ \texttt{= lqr}($\mathbf A^{\top}$, $\mathbf C^{\top}$, $\mathbf Q$, $\mathbf R$).

where $\mathbf Q$ and $\mathbf R$ are weight  matrices, determining the "sensitivity" or "aggressiveness" of the observer.


\end{flushleft}
\end{frame}




\begin{frame}{Observation and Control}
\framesubtitle{LTI}
\begin{flushleft}

Thus we get dynamics+observer combination:

\begin{equation}
\begin{cases}
\dot {\bo{x}} = \bo{A} \bo{x} + \bo{B} \bo{u} \\
\bo{u} = -\bo{K} (\bo{x} - \bo{x}^*(t)) + \bo{u}^*(t)\\
\bo{y} = \bo{C} \bo{x} \\
\hat{\dot {\mathbf x}} = \mathbf A \hat{\mathbf x} + \mathbf B \mathbf u + \mathbf L(\mathbf y - \bo{C} \hat{\mathbf x})
\end{cases}
\end{equation}

\bigskip

where $\mathbf A - \mathbf B \mathbf K < 0$ and $\mathbf A^{\top} - 
\mathbf C^{\top} \mathbf L^{\top} < 0$.


\end{flushleft}
\end{frame}


\begin{frame}{Observation and Control}
\framesubtitle{Affine case}
\begin{flushleft}

Affine case is almost the same:

\begin{equation}
\begin{cases}
\dot {\bo{x}} = \bo{A} \bo{x} + \bo{B} \bo{u} + \bo{c} \\
\bo{u} = -\bo{K} (\bo{x} - \bo{x}^*(t)) + \bo{u}^*(t)\\
\bo{y} = \bo{C} \bo{x} \\
\hat{\dot {\mathbf x}} = \mathbf A \hat{\mathbf x} + \mathbf B \mathbf u + \bo{c} + \mathbf L(\mathbf y - \bo{C} \hat{\mathbf x})
\end{cases}
\end{equation}

\bigskip

where $\mathbf A - \mathbf B \mathbf K < 0$ and $\mathbf A^{\top} - 
\mathbf C^{\top} \mathbf L^{\top} < 0$.


\end{flushleft}
\end{frame}





\begin{frame}{Thank you!}
\centerline{Lecture slides are available via Moodle.}
\bigskip
\centerline{You can help improve these slides at:}
\centerline{\href{https://github.com/SergeiSa/Control-Theory-Slides-Spring-2021}{github.com/SergeiSa/Control-Theory-Slides-Spring-2021}}
\bigskip
\centerline{Check Moodle for additional links, videos, textbook suggestions.}
\end{frame}

\end{document}
