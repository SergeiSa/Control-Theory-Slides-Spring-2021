\documentclass{beamer}

\pdfmapfile{+sansmathaccent.map}


\mode<presentation>
{
  \usetheme{Warsaw} % or try Darmstadt, Madrid, Warsaw, Rochester, CambridgeUS, ...
  \usecolortheme{seahorse} % or try seahorse, beaver, crane, wolverine, ...
  \usefonttheme{serif}  % or try serif, structurebold, ...
  \setbeamertemplate{navigation symbols}{}
  \setbeamertemplate{caption}[numbered]
} 


%%%%%%%%%%%%%%%%%%%%%%%%%%%%
% itemize settings

\definecolor{mypink}{RGB}{255, 30, 80}
\definecolor{mydarkblue}{RGB}{60, 160, 255}
\definecolor{myblue}{RGB}{240, 240, 255}
\definecolor{mygreen}{RGB}{0, 200, 0}
\definecolor{mygreen2}{RGB}{245, 255, 230}
\definecolor{mygray}{gray}{0.8}

\setbeamertemplate{itemize items}[default]

\setbeamertemplate{itemize item}{\color{mygreen}$\blacksquare$}
\setbeamertemplate{itemize subitem}{\color{mydarkblue}$\blacktriangleright$}
\setbeamertemplate{itemize subsubitem}{\color{mygray}$\blacksquare$}



\setbeamercolor{palette quaternary}{fg=white,bg=mydarkblue}
\setbeamercolor{titlelike}{parent=palette quaternary}

\setbeamercolor{palette quaternary2}{fg=black,bg=myblue}
\setbeamercolor{frametitle}{parent=palette quaternary2}



\setbeamerfont{frametitle}{size=\Large,series=\scshape}
\setbeamerfont{framesubtitle}{size=\normalsize,series=\upshape}





%%%%%%%%%%%%%%%%%%%%%%%%%%%%
% block settings

\setbeamercolor{block title}{bg=red!30,fg=black}

\setbeamercolor*{block title example}{bg=mygreen!40!white,fg=black}

\setbeamercolor*{block body example}{fg= black,
bg= mygreen2}


%%%%%%%%%%%%%%%%%%%%%%%%%%%%
% URL settings
\hypersetup{
    colorlinks=true,
    linkcolor=blue,
    filecolor=blue,      
    urlcolor=blue,
}

%%%%%%%%%%%%%%%%%%%%%%%%%%

\renewcommand{\familydefault}{\rmdefault}

\usepackage{amsmath}
\usepackage{mathtools}


\usepackage{subcaption}

\newcommand{\bo}[1] {\mathbf{#1}}
\newcommand{\R} {\mathbb{R}}
\DeclareMathOperator*{\argmin}{arg\,min}


%%%%%%%%%%%%%%%%%%%%%%%%%%%%
% code settings

\usepackage{listings}
\usepackage{color}
% \definecolor{mygreen}{rgb}{0,0.6,0}
% \definecolor{mygray}{rgb}{0.5,0.5,0.5}
\definecolor{mymauve}{rgb}{0.58,0,0.82}
\lstset{ 
  backgroundcolor=\color{white},   % choose the background color; you must add \usepackage{color} or \usepackage{xcolor}; should come as last argument
  basicstyle=\footnotesize,        % the size of the fonts that are used for the code
  breakatwhitespace=false,         % sets if automatic breaks should only happen at whitespace
  breaklines=true,                 % sets automatic line breaking
  captionpos=b,                    % sets the caption-position to bottom
  commentstyle=\color{mygreen},    % comment style
  deletekeywords={...},            % if you want to delete keywords from the given language
  escapeinside={\%*}{*)},          % if you want to add LaTeX within your code
  extendedchars=true,              % lets you use non-ASCII characters; for 8-bits encodings only, does not work with UTF-8
  firstnumber=0000,                % start line enumeration with line 0000
  frame=single,	                   % adds a frame around the code
  keepspaces=true,                 % keeps spaces in text, useful for keeping indentation of code (possibly needs columns=flexible)
  keywordstyle=\color{blue},       % keyword style
  language=Octave,                 % the language of the code
  morekeywords={*,...},            % if you want to add more keywords to the set
  numbers=left,                    % where to put the line-numbers; possible values are (none, left, right)
  numbersep=5pt,                   % how far the line-numbers are from the code
  numberstyle=\tiny\color{mygray}, % the style that is used for the line-numbers
  rulecolor=\color{black},         % if not set, the frame-color may be changed on line-breaks within not-black text (e.g. comments (green here))
  showspaces=false,                % show spaces everywhere adding particular underscores; it overrides 'showstringspaces'
  showstringspaces=false,          % underline spaces within strings only
  showtabs=false,                  % show tabs within strings adding particular underscores
  stepnumber=2,                    % the step between two line-numbers. If it's 1, each line will be numbered
  stringstyle=\color{mymauve},     % string literal style
  tabsize=2,	                   % sets default tabsize to 2 spaces
  title=\lstname                   % show the filename of files included with \lstinputlisting; also try caption instead of title
}

%%%%%%%%%%%%%%%%%%%%%%%%%%%%
% tikz settings

\usepackage{tikz}
\tikzset{every picture/.style={line width=0.75pt}}


\title{Stabilizing Control}
\subtitle{Control Theory, Lecture 4}
\author{by Sergei Savin}
\centering
\date{Spring 2021}



\begin{document}
\maketitle


\begin{frame}{Content}

\begin{itemize}
\item Changing stability
\item Stabilizing control
\item Linear control
\item Affine control
\item Error dynamics
\item Affine trajectory tracking
\item Point-to-point control
\item Pure state feedback
\item Read more
\end{itemize}

\end{frame}



\begin{frame}{Changing stability}
% \framesubtitle{O}
\begin{flushleft}

Here are two LTIs:

\begin{equation}
    \dot{x} = 2 x
\end{equation}

\begin{equation}
    \dot{x} = 2 x + u
\end{equation}

First one is unstable. Second one - we don't know until we know what $u$ is.

\bigskip

If we pick $u=0$, the result is an unstable equation. But we can also pick $u$ such that the resulting dynamics is stable, such as $u=-3x$:

\begin{equation}
    \dot{x} = 2 x + u = 2 x - 3x = -x
\end{equation}

\begin{block}{ }
So, we can use \emph{control input} $u$ to change stability of the system!
\end{block}


\end{flushleft}
\end{frame}





\begin{frame}{Stabilizing control}
% \framesubtitle{O}
\begin{flushleft}

\begin{definition}
The problem of finding control law $\bo{u}$ that make a certain solution $\bo{x}^*$ of dynamical system $\dot{\bo{x}} = \bo{f}(\bo{x}, \bo{u})$ stable is called \emph{stabilizing control problem}
\end{definition}

\bigskip

This is true for both linear and non-linear systems. But for linear systems we can get a lot more details about this problem, if we restrict our choice of control law.



\end{flushleft}
\end{frame}



\begin{frame}{Linear control}
\framesubtitle{Part 1}
\begin{flushleft}

Consider an LTI system:

\begin{equation}
    \dot{\bo{x}} = \bo{A}\bo{x} + \bo{B}\bo{u}
\end{equation}

and let us chose control as a linear function of the \emph{state} $x$:

\begin{equation}
    \bo{u} = -\bo{K}\bo{x}
\end{equation}

Thus, we know how the system is going to look when the control is applied:

\begin{equation}
    \dot{\bo{x}} = \bo{A}\bo{x} - \bo{B}\bo{K}\bo{x}
\end{equation}
\begin{equation}
\label{eq:closed_loop}
    \dot{\bo{x}} = (\bo{A} - \bo{B}\bo{K})\bo{x}
\end{equation}

Note that \eqref{eq:closed_loop} is an autonomous system. We call this (a system that was no autonomous, but became one after substituting control law) a \emph{closed loop} system.

\end{flushleft}
\end{frame}



\begin{frame}{Linear control}
\framesubtitle{Part 2}
\begin{flushleft}

Observing the system $\dot{\bo{x}} = (\bo{A} - \bo{B}\bo{K})\bo{x}$ we obtained, we can notice that we already have the tools to analyse its stability. Namely:

\begin{block}{Stability condition for LTI closed-loop system}
The real parts of the eigenvalues of the matrix $(\bo{A} - \bo{B}\bo{K})$ should negative for asymptotic stability, or non-positive for stability in the sense of Lyapunov
\end{block}


\bigskip

So, all you need to do is to find such $\bo{K}$ that $(\bo{A} - \bo{B}\bo{K})$ has eigenvalues with negative real parts, and you made a stable closed-loop system!

\end{flushleft}
\end{frame}




\begin{frame}{Affine control}
\framesubtitle{Part 1}
\begin{flushleft}

We don't have to limit ourselves to just this $\dot{\bo{x}} = \bo{A}\bo{x} + \bo{B}\bo{u}$ and $\bo{u} = -\bo{K}\bo{x}$ pair. 

\bigskip

In fact, this pair mostly works for the simple case when the solution we want to stabilize is trivial $\bo{x}^*(t) = 0$.


\end{flushleft}
\end{frame}



\begin{frame}{Affine control}
\framesubtitle{Part 2}
\begin{flushleft}

Let us consider a slightly more complicated system:

\begin{equation}
    \dot{\bo{x}} = \bo{A}\bo{x} + \bo{B}\bo{u} + \bo{c}
\end{equation}

This is called \emph{affine system}, because of the constant term $\bo{c}$. What is the control that stabilizes this system? Let us propose an \emph{affine control law}:

\begin{equation}
    \bo{u} = -\bo{K}\bo{x} + \bo{u}^*
\end{equation}

where $\bo{u}^*$ is a constant term.

\end{flushleft}
\end{frame}





\begin{frame}{Affine control}
\framesubtitle{Part 3}
\begin{flushleft}

Thus, from $\dot{\bo{x}} = \bo{A}\bo{x} + \bo{B}\bo{u} + \bo{c}$ and $\bo{u} = -\bo{K}\bo{x} + \bo{u}^*$ we get the following closed-loop system:

\begin{equation}
    \dot{\bo{x}} = \bo{A}\bo{x} - \bo{B}\bo{K}\bo{x} + \bo{B}\bo{u}^* + \bo{c}
\end{equation}

And as long as we can choose such $\bo{u}^*$ that $\bo{B}\bo{u}^* = -\bo{c}$, we will get back to the previously seen form $\dot{\bo{x}} = (\bo{A} - \bo{B}\bo{K})\bo{x}$.

\bigskip

\begin{block}{Existence of the stabilizing control}
Notice that same as it is possible that there exists no such $\bo{K}$ that $\dot{\bo{x}} = (\bo{A} - \bo{B}\bo{K})\bo{x}$ is stable, same there might exist no such $\bo{u}^*$ that $\bo{B}\bo{u}^* = -\bo{c}$
\end{block}

\end{flushleft}
\end{frame}



\begin{frame}{Error dynamics}
\framesubtitle{Part 1}
\begin{flushleft}

Let us now consider an arbitrary solution $\bo{x}^* = \bo{x}^*(t)$ for the linear system

\begin{equation}
\label{eq:ErrorDynamics_1}
    \dot{\bo{x}} = \bo{A}\bo{x} + \bo{B}\bo{u}
\end{equation}

and try to find a stabilizing control for it.

\bigskip

Our first step is to notice that, if $\bo{x}^* = \bo{x}^*(t)$ is a solution, that means that it satisfies the ODE \eqref{eq:ErrorDynamics_1}:

\begin{equation}
    \dot{\bo{x}}^* = \bo{A}\bo{x}^* + \bo{B}\bo{u}^*
\end{equation}

where $\bo{u}^* = \bo{u}^*(t)$ is some control law, for which the solution $\bo{x}^* = \bo{x}^*(t)$ is obtained. 

\end{flushleft}
\end{frame}



\begin{frame}{Error dynamics}
\framesubtitle{Part 2}
\begin{flushleft}

 If we are not given $\bo{u}^* = \bo{u}^*(t)$, we can compute it as:

\begin{equation}
    \bo{u}^* = \bo{B}^+(\dot{\bo{x}}^* - \bo{A}\bo{x}^*)
\end{equation}

where $\bo{B}^+$ is a pseudo-inverse, and the solution to this least-squared problem will have to have no residual (since $\bo{x}^* = \bo{x}^*(t)$ is a solution).

\begin{equation}
   || \dot{\bo{x}}^* - \bo{A}\bo{x}^* - \bo{B}\bo{B}^+(\dot{\bo{x}}^* - \bo{A}\bo{x}^*) || = 0
\end{equation}

\end{flushleft}
\end{frame}



\begin{frame}{Error dynamics}
\framesubtitle{Part 3}
\begin{flushleft}

Now, let us introduce the concept of \emph{control error} $\bo{e}$:

\begin{equation}
    \bo{e} = \bo{x} - \bo{x}^*
\end{equation}

\begin{block}{Control error and stability}
If control error goes to zero asymptotically, every solution goes to $\bo{x}^*$.
\end{block}

\end{flushleft}
\end{frame}





\begin{frame}{Error dynamics}
\framesubtitle{Part 4}
\begin{flushleft}

Remember that we have two simultaneous equations: $\dot{\bo{x}} = \bo{A}\bo{x} + \bo{B}\bo{u}$ and $\dot{\bo{x}}^* = \bo{A}\bo{x}^* + \bo{B}\bo{u}^*$. We can now subtract one from the other to get:

\begin{equation}
    \dot{\bo{x}} - \dot{\bo{x}}^* = \bo{A}\bo{x} - \bo{A}\bo{x}^* + \bo{B}\bo{u} - \bo{B}\bo{u}^*
\end{equation}

in other words:

\begin{equation}
    \dot{\bo{e}} = \bo{A}\bo{e} + \bo{B}\bo{v}
\end{equation}

where $\bo{v} = \bo{u} - \bo{u}^*$

\end{flushleft}
\end{frame}




\begin{frame}{Error dynamics}
\framesubtitle{Part 5}
\begin{flushleft}

We arrived at a new dynamical system $\dot{\bo{e}} = \bo{A}\bo{e} + \bo{B}\bo{v}$, which is an LTI, and we are interested in stabilizing the solution $\bo{e}^* = 0$. We know how to do it with a linear control law:

\begin{equation}
    \bo{v} = -\bo{K}\bo{e}
\end{equation}

Now remember that $\bo{v} = \bo{u} - \bo{u}^*$ and $\bo{e} = \bo{x} - \bo{x}^*$, this will become:

\begin{equation}
    \bo{u} = -\bo{K}(\bo{x} - \bo{x}^*) + \bo{u}^*
\end{equation}

\end{flushleft}
\end{frame}



\begin{frame}{Error dynamics}
\framesubtitle{Part 6}
\begin{flushleft}

This control law $\bo{u} = -\bo{K}(\bo{x} - \bo{x}^*) + \bo{u}^*$ can be thought of as consisting of two parts:

\begin{itemize}
    \item Feedback control $\bo{u}_{FB} = -\bo{K}(\bo{x} - \bo{x}^*)$, which depends on the control error (which requires a feedback about the current state of your system)
    \item Feed-forward control $\bo{u}_{FF} = \bo{u}^*$, which depends only on the trajectory and the equations of dynamics of your system, but not on your current state
\end{itemize}

\end{flushleft}
\end{frame}




\begin{frame}{Affine trajectory tracking}
\framesubtitle{Part 1}
\begin{flushleft}

What we just did - stabilization of the arbitrary trajectory $\bo{x}^* = \bo{x}^*(t)$ - is also called \emph{trajectory tracking control}, or \emph{trajectory stabilization}. The solution to stabilize is called \emph{trajectory}.

\bigskip

Just for completeness, let's consider the system:

\begin{equation}
\label{eq:AffineErrorDynamics_1}
    \dot{\bo{x}} = \bo{A}\bo{x} + \bo{B}\bo{u} + \bo{c}
\end{equation}

and stabilize trajectory $\bo{x}^* = \bo{x}^*(t)$.

\end{flushleft}
\end{frame}




\begin{frame}{Affine trajectory tracking}
\framesubtitle{Part 2}
\begin{flushleft}

We start by observing that, as before, our solution gives us equality:

\begin{equation}
\label{eq:AffineErrorDynamics_2}
    \dot{\bo{x}}^* = \bo{A}\bo{x}^* + \bo{B}\bo{u}^* + \bo{c}
\end{equation}

and after introducing control error and subtracting \eqref{eq:AffineErrorDynamics_2} from the original dynamics \eqref{eq:AffineErrorDynamics_1}, we get:

\begin{equation}
    \dot{\bo{e}} = \bo{A}\bo{e} + \bo{B}\bo{v}
\end{equation}

where $\bo{v} = \bo{u} - \bo{u}^*$, which we already saw before. The only difference is that now $\bo{u}^*$ is found as:

\begin{equation}
    \bo{u}^* = \bo{B}^+(\dot{\bo{x}}^* - \bo{A}\bo{x}^* - \bo{c})
\end{equation}

\end{flushleft}
\end{frame}




\begin{frame}{Point-to-point control}
\framesubtitle{Part 1}
\begin{flushleft}

What if we want to move our system $\dot{\bo{x}} = \bo{A}\bo{x} + \bo{B}\bo{u}$ from initial condition to some desired state $\bo{x}^*$. The difference is that we do not have a solution that we used before, only the desired \emph{node} is given. This can be called \emph{point-to-point control}

\bigskip

Let us start by giving the form of the control law:

\begin{equation}
    \bo{u} = -\bo{K}(\bo{x} - \bo{x}^*) + \bo{u}^*
\end{equation}

And thus we can re-write the dynamics as:

\begin{equation}
\label{eq:dynamics_point_to_point}
    \dot{\bo{x}} = \bo{A}\bo{x} - \bo{B}\bo{K}(\bo{x} - \bo{x}^*) + \bo{B}\bo{u}^*
\end{equation}

\end{flushleft}
\end{frame}




\begin{frame}{Point-to-point control}
\framesubtitle{Part 2}
\begin{flushleft}

Let us consider how the system $\dot{\bo{x}} = \bo{A}\bo{x} - \bo{B}\bo{K}(\bo{x} - \bo{x}^*) + \bo{B}\bo{u}^*$ will behave at the point $\bo{x}^*$. We know that $\dot{\bo{x}}^* = 0$:

\begin{equation}
    0 = \bo{A}\bo{x}^* - \bo{B}\bo{K}(\bo{x}^* - \bo{x}^*) + \bo{B}\bo{u}^*
\end{equation}
\begin{equation}
\label{eq:solution_point_to_point}
    0 = \bo{A}\bo{x}^* + \bo{B}\bo{u}^*
\end{equation}

which we saw before. It provides us solution to the forward dynamics:

\begin{equation}
    \bo{u}^* = -\bo{B}^+\bo{A}\bo{x}^*
\end{equation}

Subtracting solution \eqref{eq:solution_point_to_point} from the original dynamics \eqref{eq:dynamics_point_to_point}, we get familiar error dynamics $\dot{\bo{e}} = (\bo{A} - \bo{B}\bo{K})\bo{e}$.

\end{flushleft}
\end{frame}




\begin{frame}{Pure state feedback}
\framesubtitle{Part 1}
\begin{flushleft}

Given $\dot{\bo{x}} = \bo{A}\bo{x} + \bo{B}\bo{u}$ and the desired state $\bo{x}^*$ we can do a point-to-point control with the following \emph{pure state feedback control}:

\begin{equation}
    \bo{u} = -\bo{K}\bo{x} + \bo{u}^*
\end{equation}

We can re-write the dynamics as:

\begin{equation}
    \dot{\bo{x}} = \bo{A}\bo{x} - \bo{B}\bo{K}\bo{x} + \bo{B}\bo{u}^*
\end{equation}

As before, we know that at the node,  $\dot{\bo{x}}^* = 0$:

\begin{equation}
    0 = (\bo{A} - \bo{B}\bo{K})\bo{x}^* + \bo{B}\bo{u}^*
\end{equation}

Thus we can solve for $\bo{u}^*$

\begin{equation}
    \bo{u}^* = -\bo{B}^+(\bo{A} - \bo{B}\bo{K})\bo{x}^*
\end{equation}

\end{flushleft}
\end{frame}



\begin{frame}{Pure state feedback}
\framesubtitle{Part 2}
\begin{flushleft}

The rest is the same. Error dynamics is $\dot{\bo{e}} = (\bo{A} - \bo{B}\bo{K})\bo{e}$.

\bigskip

Note that when $\bo{u} = -\bo{K}\bo{x} + \bo{u}^*$, we got feed-forward control in the form:

\[
    \bo{u}^* = -\bo{B}^+(\bo{A} - \bo{B}\bo{K})\bo{x}^*
\]

\bigskip

But when we had $\bo{u} = -\bo{K}(\bo{x} - \bo{x}^*) + \bo{u}^*$, our feed-forward control was

\[
    \bo{u}^* = -\bo{B}^+\bo{A}\bo{x}^*
\]

\bigskip

The difference has to do with how the two control methods behave at the node.

\end{flushleft}
\end{frame}





\begin{frame}{Read more}

\begin{itemize}
\item Richard M. Murray Control and Dynamical Systems California Institute of Technology \href{http://www.cds.caltech.edu/~murray/books/AM08/pdf/obc-trajgen_03Jan10.pdf}{Optimization-Based Control}
\item \href{https://apmonitor.com/pdc/index.php/Main/ModelSimulation}{Dynamic Simulation in Python}


\end{itemize}

\end{frame}



\begin{frame}{Thank you!}
\centerline{Lecture slides are available via Moodle.}
\bigskip
\centerline{You can help improve these slides at:}
\centerline{\href{https://github.com/SergeiSa/Control-Theory-Slides-Spring-2021}{github.com/SergeiSa/Control-Theory-Slides-Spring-2021}}
\bigskip
\centerline{Check Moodle for additional links, videos, textbook suggestions.}
\end{frame}

\end{document}
