\documentclass{beamer}

\pdfmapfile{+sansmathaccent.map}


\mode<presentation>
{
  \usetheme{Warsaw} % or try Darmstadt, Madrid, Warsaw, Rochester, CambridgeUS, ...
  \usecolortheme{seahorse} % or try seahorse, beaver, crane, wolverine, ...
  \usefonttheme{serif}  % or try serif, structurebold, ...
  \setbeamertemplate{navigation symbols}{}
  \setbeamertemplate{caption}[numbered]
} 


%%%%%%%%%%%%%%%%%%%%%%%%%%%%
% itemize settings

\definecolor{mypink}{RGB}{255, 30, 80}
\definecolor{mydarkblue}{RGB}{60, 160, 255}
\definecolor{myblue}{RGB}{240, 240, 255}
\definecolor{mygreen}{RGB}{0, 200, 0}
\definecolor{mygreen2}{RGB}{245, 255, 230}
\definecolor{mygray}{gray}{0.8}

\setbeamertemplate{itemize items}[default]

\setbeamertemplate{itemize item}{\color{mygreen}$\blacksquare$}
\setbeamertemplate{itemize subitem}{\color{mydarkblue}$\blacktriangleright$}
\setbeamertemplate{itemize subsubitem}{\color{mygray}$\blacksquare$}



\setbeamercolor{palette quaternary}{fg=white,bg=mydarkblue}
\setbeamercolor{titlelike}{parent=palette quaternary}

\setbeamercolor{palette quaternary2}{fg=black,bg=myblue}
\setbeamercolor{frametitle}{parent=palette quaternary2}



\setbeamerfont{frametitle}{size=\Large,series=\scshape}
\setbeamerfont{framesubtitle}{size=\normalsize,series=\upshape}





%%%%%%%%%%%%%%%%%%%%%%%%%%%%
% block settings

\setbeamercolor{block title}{bg=red!30,fg=black}

\setbeamercolor*{block title example}{bg=mygreen!40!white,fg=black}

\setbeamercolor*{block body example}{fg= black,
bg= mygreen2}


%%%%%%%%%%%%%%%%%%%%%%%%%%%%
% URL settings
\hypersetup{
    colorlinks=true,
    linkcolor=blue,
    filecolor=blue,      
    urlcolor=blue,
}

%%%%%%%%%%%%%%%%%%%%%%%%%%

\renewcommand{\familydefault}{\rmdefault}

\usepackage{amsmath}
\usepackage{mathtools}


\usepackage{subcaption}

\newcommand{\bo}[1] {\mathbf{#1}}
\newcommand{\R} {\mathbb{R}}


%%%%%%%%%%%%%%%%%%%%%%%%%%%%
% code settings

\usepackage{listings}
\usepackage{color}
% \definecolor{mygreen}{rgb}{0,0.6,0}
% \definecolor{mygray}{rgb}{0.5,0.5,0.5}
\definecolor{mymauve}{rgb}{0.58,0,0.82}
\lstset{ 
  backgroundcolor=\color{white},   % choose the background color; you must add \usepackage{color} or \usepackage{xcolor}; should come as last argument
  basicstyle=\footnotesize,        % the size of the fonts that are used for the code
  breakatwhitespace=false,         % sets if automatic breaks should only happen at whitespace
  breaklines=true,                 % sets automatic line breaking
  captionpos=b,                    % sets the caption-position to bottom
  commentstyle=\color{mygreen},    % comment style
  deletekeywords={...},            % if you want to delete keywords from the given language
  escapeinside={\%*}{*)},          % if you want to add LaTeX within your code
  extendedchars=true,              % lets you use non-ASCII characters; for 8-bits encodings only, does not work with UTF-8
  firstnumber=0000,                % start line enumeration with line 0000
  frame=single,	                   % adds a frame around the code
  keepspaces=true,                 % keeps spaces in text, useful for keeping indentation of code (possibly needs columns=flexible)
  keywordstyle=\color{blue},       % keyword style
  language=Octave,                 % the language of the code
  morekeywords={*,...},            % if you want to add more keywords to the set
  numbers=left,                    % where to put the line-numbers; possible values are (none, left, right)
  numbersep=5pt,                   % how far the line-numbers are from the code
  numberstyle=\tiny\color{mygray}, % the style that is used for the line-numbers
  rulecolor=\color{black},         % if not set, the frame-color may be changed on line-breaks within not-black text (e.g. comments (green here))
  showspaces=false,                % show spaces everywhere adding particular underscores; it overrides 'showstringspaces'
  showstringspaces=false,          % underline spaces within strings only
  showtabs=false,                  % show tabs within strings adding particular underscores
  stepnumber=2,                    % the step between two line-numbers. If it's 1, each line will be numbered
  stringstyle=\color{mymauve},     % string literal style
  tabsize=2,	                   % sets default tabsize to 2 spaces
  title=\lstname                   % show the filename of files included with \lstinputlisting; also try caption instead of title
}

%%%%%%%%%%%%%%%%%%%%%%%%%%%%
% tikz settings

\usepackage{tikz}
\tikzset{every picture/.style={line width=0.75pt}}


\title{Lyapunov Theory, Lyapunov equations}
\subtitle{Control Theory, Lecture 9}
\author{by Sergei Savin}
\centering
\date{Spring 2021}



\begin{document}
\maketitle


\begin{frame}{Content}

\begin{itemize}
\item Lyapunov method: stability criteria
\item Lyapunov method: examples
\item Linear case
\item Discrete case
\item Lyapunov equations
% \item Read more
\end{itemize}

\end{frame}





\begin{frame}{Lyapunov method: stability criteria}
% \framesubtitle{Parameter estimation}
\begin{flushleft}

\begin{block}{Asymptotic stability criteria}
Autonomous dynamic system $\dot{\bo{x}} = \bo{f}(\bo{x})$ is assymptotically stable, if there exists a scalar function $V = V(\bo{x}) > 0$, whose time derivative is negative $\dot V(\bo{x}) < 0$, except $V(\bo{0}) = 0$, $\dot V(\bo{0}) = 0$.
\end{block}

\begin{block}{Marginal stability criteria}
$\dot{\bo{x}} = \bo{f}(\bo{x})$ is stable in the sense of Lyapunov, $\exists V = V(\bo{x}) \geq 0$, $\dot V(\bo{x}) \leq 0$.
\end{block}

\begin{definition}
Function $V = V(\bo{x}) > 0$ in this case is called \emph{Lyapunov function}.
\end{definition}

\bigskip

This is not the only type of stability as you remember, you are invited to study criteria for other stability types on your own.

\end{flushleft}
\end{frame}


\begin{frame}{Lyapunov method: examples}
\framesubtitle{Example 1}
\begin{flushleft}

Take dynamical system $\dot{x} = -x$. 

\bigskip

We propose a \emph{Lyapunov function candidate} $V(x) = x^2 \geq 0$. Let's find its derivative:

\begin{equation}
    \dot V(x) = \frac{\partial V}{\partial x} (-x) = 2x (-x) = -x^2 \leq 0
\end{equation}

This satisfies the Lyapunov criteria, so the system is stable. It is in fact asymptotically stable, because $V(x) \neq 0$ if $x \neq 0$.

\end{flushleft}
\end{frame}



\begin{frame}{Lyapunov method: examples}
\framesubtitle{Example 2}
\begin{flushleft}

Consider oscillator $\ddot{q} = f(q, \dot{q}) = -\dot{q}$. 

\bigskip

We propose a \emph{Lyapunov function candidate} $V(q, \dot{q}) = T(q, \dot{q}) = \frac{1}{2} \dot{q}^2 \geq 0$, where $T(q, \dot{q})$ is kinetic energy of the system. Let's find its derivative:

\begin{equation}
    \dot V(q, \dot{q}) = 
    \frac{\partial V}{\partial q}       \dot{q} +
    \frac{\partial V}{\partial \dot{q}} f(q, \dot{q}) = 
    \dot{q} (-\dot{q}) = -\dot{q}^2 \leq 0
\end{equation}


This satisfies the Lyapunov criteria, so the system is stable. It is in fact not proven to be asymptotically stable, because $V(q, \dot{q}) = 0$ for any $q$ as long as $\dot{q} = 0$, and $\dot V(q, \dot{q}) = 0$ for any $q$ as long as $\dot{q} = 0$.

\end{flushleft}
\end{frame}




\begin{frame}{Lyapunov method: examples}
\framesubtitle{Example 3}
\begin{flushleft}

Consider pendulum $\ddot{q} = f(q, \dot{q}) = -\dot{q} - sin(q)$. 

\bigskip

We propose a \emph{Lyapunov function candidate} $V(q, \dot{q}) = E(q, \dot{q}) = \frac{1}{2} \dot{q}^2 + 1 - cos(q)\geq 0$, where $E(q, \dot{q})$ is total energy of the system. Let's find its derivative:

\begin{equation}
    \dot V(q, \dot{q}) = 
    \frac{\partial V}{\partial q}       \dot{q} +
    \frac{\partial V}{\partial \dot{q}} f(q, \dot{q}) = 
    \dot{q} sin(q) + \dot{q}(-\dot{q} - sin(q)) =
    -\dot{q}^2 \leq 0
\end{equation}


This satisfies the Lyapunov criteria, so the system is stable. It is in fact asymptotically stable, because $V(q, \dot{q}) \neq 0$ if $q \neq 0$ and $\dot{q} \neq 0$.

\end{flushleft}
\end{frame}


\begin{frame}{Linear case}
\framesubtitle{Part 1}
\begin{flushleft}

As you saw, Lyapunov method allows you to deal with nonlinear systems, as well as linear ones. But for linear ones there are additional properties we can use.

\begin{block}{Observation 1}
For a linear system $\dot{\bo{x}} = \bo{A}\bo{x}$ we can always pick Lyapunov function candidate in the form $V = \bo{x}^\top\bo{S}\bo{x} \geq 0$, where $\bo{S}$ is a positive semidefinite matrix.
\end{block}

\bigskip

Next slides will shows where this leads us.

\end{flushleft}
\end{frame}


\begin{frame}{Linear case}
\framesubtitle{Part 2}
\begin{flushleft}

Given $\dot{\bo{x}} = \bo{A}\bo{x}$ and $V = \bo{x}^\top\bo{S}\bo{x} \geq 0$, let's find its derivative:

\begin{equation}
    \dot V(\bo{x}) = \dot{\bo{x}}^\top\bo{S}\bo{x} + 
    \bo{x}^\top\bo{S}\dot{\bo{x}}
\end{equation}

\begin{equation}
    \dot V(\bo{x}) = (\bo{A}\bo{x})^\top\bo{S}\bo{x} + 
    \bo{x}^\top\bo{S}\bo{A}\bo{x} = 
    \bo{x}^\top(\bo{A}^\top\bo{S} + \bo{S}\bo{A})\bo{x}
\end{equation}

Notice that $\dot V(x)$ should be negative for all $\bo{x}$ for the system to be stable, meaning that $\bo{A}^\top\bo{S} + \bo{S}\bo{A}$ should be negative semidefinite. A more strict form of this requirement is \emph{Lyapunov equation}:

\begin{equation}
    \bo{A}^\top\bo{S} + \bo{S}\bo{A} = -\bo{Q}
\end{equation}

where $\bo{Q}$ is a positive-definite matrix.

\end{flushleft}
\end{frame}



\begin{frame}{Discrete case}
\framesubtitle{Part 1}
\begin{flushleft}

\begin{block}{Marginal stability criteria, discrete case}
Given $\bo{x}_{i+1} = \bo{f}(\bo{x}_i)$, if $V(\bo{x}_i) \geq 0$, and $V(\bo{x}_{i+1}) - V(\bo{x}_i) \leq 0$, the system is stable.
\end{block}

\bigskip 

Same as before, for linear systems we will be choosing \emph{positive semidefinite quadratic forms} as Lyapunov function candidates.

\end{flushleft}
\end{frame}



\begin{frame}{Discrete case}
\framesubtitle{Part 2}
\begin{flushleft}

Consider dynamics $\bo{x}_{i+1} = \bo{A}\bo{x}_i$ and $V = \bo{x}_i^\top\bo{S}\bo{x}_i \geq 0$, let's find $V(\bo{x}_{i+1}) - V(\bo{x}_i)$:

\begin{equation}
    V(\bo{x}_{i+1}) - V(\bo{x}_i) = (\bo{A}\bo{x}_i)^\top\bo{S}\bo{A}\bo{x}_i - 
    \bo{x}_i^\top\bo{S}\bo{x}_i
\end{equation}
\begin{equation}
    V(\bo{x}_{i+1}) - V(\bo{x}_i) = \bo{x}_i^\top(\bo{A}^\top\bo{S}\bo{A} - \bo{S})\bo{x}_i
\end{equation}

Notice that $V(\bo{x}_{i+1}) - V(\bo{x}_i)$ should be negative for all $\bo{x}_i$ for the system to be stable, meaning that $\bo{A}^\top\bo{S}\bo{A} - \bo{S}$ should be negative semidefinite. A more strict form of this requirement is \emph{Discrete Lyapunov equation}:

\begin{equation}
    \bo{A}^\top\bo{S}\bo{A} - \bo{S} = -\bo{Q}
\end{equation}

where $\bo{Q}$ is a positive-definite matrix.

\end{flushleft}
\end{frame}






\begin{frame}{Lyapunov equations}
% \framesubtitle{Local coordinates}
\begin{flushleft}

In practice, you can easily use Lyapunov equations for stability verification. Python and MATLAB have built-in functionality to solve it:

\begin{itemize}
    \item scipy: \texttt{linalg.solve\_continuous\_lyapunov(A, Q)}
    \item MATLAB: \texttt{lyap(A,Q)}
\end{itemize}

\end{flushleft}
\end{frame}




\begin{frame}{Thank you!}
\centerline{Lecture slides are available via Moodle.}
\bigskip
\centerline{You can help improve these slides at:}
\centerline{\href{https://github.com/SergeiSa/Control-Theory-Slides-Spring-2021}{github.com/SergeiSa/Control-Theory-Slides-Spring-2021}}
\bigskip
\centerline{Check Moodle for additional links, videos, textbook suggestions.}
\end{frame}

\end{document}
